\documentclass[12pt, letterpaper, twoside]{article}
\usepackage[utf8]{inputenc}
\usepackage{graphicx}
\usepackage{amsmath}
\usepackage{amsfonts}
\begin{document}
\title{MAT3360 Oblig 1}
\author{Espen Lønes}
\date{\today}
\maketitle
\ \\
Question 1)\\
a)\\
\ \\
Prove that
$\int_0^1 u(x)^2 dx \leq \frac{1}{2} \int_0^1 u'(x)^2 dx$\\
Where $u \in C_0^2((0,1))$ and $x \in (0,1)$. We start with.
$$
u(x) = \int_0^x u'(y) dy 
$$
And use the Cauchy-Schwarts inequality.
$$
|u(x)| = |\int_0^x u'(y) dy| \leq (\int_0^x 1^2 dy)^{\frac{1}{2}}(\int_0^x u'(y)^2 dy)^{\frac{1}{2}}
$$
Square on both sides
$$
u(x)^2 \leq x(\int_0^x u'(y)^2 dy)
$$
Integrate from 0 to 1.
$$
\int_0^1 u(x)^2 dx \leq \int_0^1 \int_0^x xu'(y)^2 dy\ dx
$$
Change integration order.
$$
\int_0^1 \int_y^1 xu'(y)^2 dx\ dy
=
\int_0^1 \frac{1}{2} u'(y)^2 - \frac{1}{2} y^2 u'(y)^2\ dy
$$ 
$
- \frac{1}{2} y^2 u'(y)^2 \leq 0 
$
\\
Therefore. 
$$
\int_0^1 \frac{1}{2} u'(y)^2 - \frac{1}{2} y^2 u'(y)^2\ dy
\leq
\int_0^1 \frac{1}{2} u'(y)^2 dy
=
\frac{1}{2} \int_0^1 u'(x)^2 dx
$$
and
$$
\int_0^1 u(x)^2 dx \leq \frac{1}{2} \int_0^1 u'(x)^2 dx
$$
\ \\
b)\\
Prove
$$
h\sum_{j=1}^n v_j^2 \leq \frac{1}{2} h \sum_{j=1}^{n+1} (D_-^h v_j)^2
$$
Where $v \in D_{0,h}$. We start with.
$$
v_j = h\sum_{k=1}^j (D_-^h v_k)
$$
We then use the Cauchy-Schwarts inequality.
$$
|h\sum_{k=1}^j (D_-^h v_k)| \leq (h\sum_{k=1}^j 1^2)^{\frac{1}{2}}(h \sum_{k=1}^j (D_-^h v_k)^2)^{\frac{1}{2}}
$$
Square on both sides.
$$
v_j^2 \leq (h\sum_{k=1}^j 1^2)(h \sum_{k=1}^j (D_-^h v_k)^2)
=
(hj)(h \sum_{k=1}^j (D_-^h v_k)^2)
$$
Then sum from $j=1$ to $j=n$ and multiply by h on both sides.
$$
h\sum_{j=1}^n v_j^2 \leq h^3 \sum_{j=1}^n \sum_{k=1}^j j(D_-^h v_k)^2
$$
Change summation order.
$$
h\sum_{j=1}^n v_j^2 \leq h^3 \sum_{k=1}^{n+1} \sum_{j=k}^n j(D_-^h v_k)^2
$$
$$
\sum_{j=k}^{n+1} j \leq \sum_{j=1}^n j = \frac{n(n+1)}{2}
$$
Inserting this gives us.
$$
h\sum_{j=1}^n v_j^2 \leq h^3 \sum_{k=1}^{n+1} \frac{n(n+1)}{2} (D_-^h v_k)^2
$$
Then using the fact that $h=\frac{1}{(n+1)}$
$$
h\sum_{j=1}^n v_j^2 \leq h \frac{1}{(n+1)^2} \frac{n(n+1)}{2} \sum_{k=1}^{n+1} (D_-^h v_k)^2
=
\frac{1}{2} h \frac{n}{(n+1)} \sum_{k=1}^n (D_-^h v_k)^2
$$
\ \\
$$
lim_{n \to \infty} \frac{n}{(n+1)} = 1
$$ 
So
$$
\frac{n}{(n+1)} \leq 1
$$
Giving us.
$$
\frac{1}{2} h \frac{n}{(n+1)} \sum_{k=1}^{n+1} (D_-^h v_k)^2 
\leq
\frac{1}{2} h  \sum_{k=1}^{n+1} (D_-^h v_k)^2 
=
\frac{1}{2} h  \sum_{j=1}^{n+1} (D_-^h v_j)^2 
$$
Giving
$$
h\sum_{j=1}^n v_j^2 \leq \frac{1}{2} h \sum_{j=1}^{n+1} (D_-^h v_j)^2
$$
\ \\
\newpage
Question 2)\\
\ \\
We have $-u''(x) + u(x) = f(x)$ \ \ \ \ (1) \\
$u \in C_0^2((0,1))$ and $x \in (0,1)$ and f is given.
\ \\
A solution $u$ is on the form.
$$
u(x) = u_h(x) + u_p(x) 
$$
Where $u_h(x)$ is the solution to the corresponding homogeneous equation:
$-u''(x) + u(x) = 0$\\
And $u_p(x)$ is a particular solution.\\
We can find the form of $u_h(x)$ by observing that
$$
u'(x) - u''(x) = 0 \iff u''(x) = u'(x) \iff u¨'(x) = C_1 e^x 
$$
Giving us.
$$
u_h(x) = C_1e^x + C_2
$$
Since we are given $f(x)=1$, $u_p(x)$ might be on the form $u_p(x) = Ax$.\\
Giving $u_p'(x) = A$ and $u_p''(x) = 0$. We must then find $A$ s.t.
$$
u'(x) - u''(x) = 1
$$
$$
A - 0 = 1 \iff A=1
$$
We then have.
$$
u(x) = C_1e^x + C_2 + x 
$$
Inserting for $u(0)= 0$ and $u(1)=0$
$$
u(0) = C_1 + C_2 = 0 \iff C_2 = -C_1
$$
$$
u(1) = C_1e - C_1 = -1 \iff C_1 = \frac{1}{e-1} \iff C_2 = -\frac{1}{e-1}
$$
Inserting $C_1$ and $C_2$ gives.
$$
u(x) = \frac{1}{1-e}e^x - \frac{1}{1-e} + x 
=
\frac{e^x - 1}{1 - e} + x
$$
\ \\
\newpage
\ \\
Question 3)\\
a)\\
\ \\
Show that if $u$ solves (1) then
$$
\int_0^1 u'(x)^2 dx = \int_0^1 f(x)u(x) dx
$$
we start with (1) and multiply by $u$ then integrate from 0 to 1. which gives.
$$
\int_0^1 u(x)u'(x) - u(x)u''(x) dx = \int_0^1 f(x)u(x) dx
$$
Left side:
$$
\int_0^1 u(x)u'(x) - u(x)u''(x) dx
=
\int_0^1 u(x)u'(x) dx - \int_0^1 u(x)u''(x) dx
$$
$$
=
\int_0^1 u(x)u'(x) dx - \left[ u(x)u'(x)|_0^1 - \int_0^1 u'(x)u'(x) dx \right]
$$
$$
=
\int_0^1 u(x)u'(x) dx + \int_0^1 u'(x)^2 dx 
$$
We now look at
$$
I = \int_0^1 u(x)u'(x) dx = u(x)u(x)|_0^1 - \int u'(x)u(x) dx = - \int u'(x)u(x) dx = -I
$$
$I = -I \iff I = 0$
So we have
$$
\int_0^1 u'(x)^2 dx = \int_0^1 f(x)u(x) dx
$$
\newpage
\ \\
b)\\
\ \\
Show
$$
\int_0^1 u(x)^2 dx \leq \frac{1}{3} \int_0^1 f(x)^2 dx
$$
We start by combining the results from 1a: $\int_0^1 u(x)^2 dx \leq \frac{1}{2} \int_0^1 u'(x)^2 dx$ and 3a: $\int_0^1 u'(x)^2 dx = \int_0^1 f(x)u(x) dx$, to get.
$$
\int_0^1 u(x)^2 dx \leq \frac{1}{2} \int_0^1 f(x)u(x) dx
$$
We use Cauchy-Schwartz on the rhs. to get.
$$
\int_0^1 u(x)^2 dx \leq \frac{1}{2} \left( \int_0^1 f(x)^2 dx \right)^2 \left( \int_0^1 u(x)^2 dx \right)^2
$$
We then use the fact that $|ab| \leq \frac{a^2}{2} + \frac{b^2}{2}$ to get.
$$
\frac{1}{2} \left( \int_0^1 f(x)^2 dx \right)^2 \left( \int_0^1 u(x)^2 dx \right)^2
\leq
\frac{1}{2} \left( \frac{1}{2} \int_0^1 f(x)^2 dx + \frac{1}{2} u(x)^2 dx \right)
$$
So
$$
\int_0^1 u(x)^2 dx \leq \frac{1}{4} \int_0^1 f(x)^2 dx + \frac{1}{4} u(x)^2 dx
$$
$$
\iff 4\int_0^1 u(x)^2 dx - \int_0^1 u(x)^2 dx \leq \int_0^1 f(x)^2 dx
$$
$$
\iff \int_0^1 u(x)^2 dx \leq \frac{1}{3} \int_0^1 f(x)^2 dx
$$
\newpage
\ \\
Question 4)\\
a)\\
\ \\
First we look at $|D_c^h u(x) - u'(x)|$.
$$
|D_c^h u(x) - u'(x)| = \frac{u(x+h) - u(x-h)}{2h} - u'(x)
$$
\begin{align}
=\frac{1}{2h} &[u(x) + hu'(x) + \frac{h^2}{2} u''(x) + \frac{h^3}{6} u'''(x) + c_1 h^4 \nonumber \\
              & -u(x) + hu'(x) - \frac{h^2}{2} u''(x) + \frac{h^3}{6} u'''(x) - c_2 h^4 ] - u'(x) \nonumber
\end{align}
$$
= u'(x) + \frac{h^2}{6} u'''(x) + c_3h^3 - u'(x)  =  \frac{h^2}{6} u'''(x) + c_3h^3
$$
Where $c_3 = \frac{c_1 - c_2}{2}$.
We see that $h \leq 1$.\\
If we give
$
a = sup_{x \in [0,1]} |u'''(x)|
$
We get.
$$
|D_c^h u(x) - u'(x)| \leq |h^2 \frac{a}{6} + c_3 h^3|
$$
Use $h^2 \leq h^3$
$$
|D_c^h u(x) - u'(x)| \leq h^2 |\frac{a}{6} + c_3| = O(h^2)
$$
Since $\frac{a}{6} + c_3$ is constant.\\
\ \\
Then we look at $|D_+^h D_-^h u(x) - u''(x)|$
$$
D_+^h D_-^h u(x) - u''(x)
=
D_+^h \left( \frac{u(x) - u(x-h)}{h} \right) - u''(x)
$$
$$
= \frac{1}{h} \left( D_+^h u(x) - D_+^h u(x-h) \right) - u''(x)
$$
$$
= \frac{1}{h} \left( \frac{u(x+h) - u(x)}{h} - \frac{u(x) - u(x - h)}{h} \right) - u''(x)
$$
$$
= \frac{u(x+h) -2u(x) + u(x-h)}{h^2} - u''(x)
$$
\newpage
\ \\
\begin{align}
= \frac{1}{h^2} &[u(x) + hu'(x) + \frac{h^2}{2} u''(x) + \frac{h^3}{6} u'''(x) + c_1h^4 \nonumber \\
                &-u(x) + hu'(x) - \frac{h^2}{2} u''(x) + \frac{h^3}{6} u'''(x) - c_2h^4 - 2u(x)] \nonumber
                - u''(x)
\end{align}
$$
= \frac{2}{h} u'(x) + \frac{h}{3} u'''(x) - h^2 c_3 - u''(x)
$$
Where the constant $c_3 = c_1 - c_2$\\
We define $a = sup_{x \in [0,1]} \frac{2}{h} u'(x)$, $b  = sup_{x \in [0,1]} \frac{h}{3} u'''(x)$ and
$c = sup_{x \in [0,1]} u''(x)$\\
And get.
$$
|D_+^h D_-^h u(x) - u''(x)| \leq |a + b - h^2c_3 - c| = O(h^2)
$$
\ \\
b)\\
\ \\
(1) is $f(x) = -u''(x) + u'(x)$\\
(2) is $-D_+^h D_-^h v_j + D_c^h v_j = f(x)$\\
And $u(x_j) = v_j$\\
We can then look at.
$$
|-D_+^h D_-^h v_j + D_c^h v_j - f(x)| 
=
|-D_+^h D_-^h u(x_j) + D_c^h u(x_j) + u''(x_j) - u'(x_j)| 
$$
We see that $ |-D_+^h D_-^h u(x_j) + u''(x_j)| = |D_+^h D_-^h u(x_j) - u''(x_j)|$.\\
Using this and $|a+b| \leq |a| + |b|$. We then get.
$$
|D_+^h D_-^h u(x_j) - u''(x_j) + D_c^h u(x_j) - u'(x_j)| 
\leq 
|D_+^h D_-^h u(x_j) - u''(x_j)| + |D_c^h u(x_j) - u'(x_j)|
$$
$$
= O(h^2)
$$
So (2) approaches (1) proportional to $h^2$. Which is a reasonable approximation. 
\newpage
\ \\
c)\\
\ \\
Find an $n \times n$ matrix $A$ and a vector $b \in \mathbb{R}^n$ s.t. (2) can be written $Av=b$\\
$$
b=
\begin{pmatrix}
f(x_1)\\
f(x_2)\\
\vdots\\
f(x_{n-1})\\
f(x_n)
\end{pmatrix}
=
\begin{pmatrix}
0\\
f(x_2)\\
\vdots\\
f(x_{n-1})\\
0
\end{pmatrix}
$$
$$
v=
\begin{pmatrix}
v_1\\
v_2\\
\vdots\\
v_{n-1}\\
v_n
\end{pmatrix}
=
\begin{pmatrix}
0\\
v_2\\
\vdots\\
v_{n-1}\\
0
\end{pmatrix}
$$
\ \\
We have $D_c^h v_j = \frac{v_{j+1} - v_{j-1}}{2h}$\\
\ \\
and $D_+^h D_-^h v_j = \frac{v_{j+1} -2v_j + v_{j-1}}{h^2}$\\
Giving us that.
$$
(2) \rightarrow - D_+^h D_-^h v_j + D_c^h v_j 
= 
- \frac{v_{j+1} -2v_j + v_{j-1}}{h^2} + \frac{v_{j+1} - v_{j-1}}{2h}
$$
\ \\
$$
=
\frac{-2v_{j+1} + 4v_j - 2v_{j-1} + hv_{j+1} - hv_{j-1}}{2h^2}
=
\frac{v_{j-1}(-2-h) + 4v_j + v_{j+1}(h - 2)}{2h^2}
$$
Giving us.\\
$$
A = \frac{1}{2h^2}
\begin{bmatrix}
1 & 0 & 0 & 0 & \dots\\
(-2-h) & 4 & (h - 2) & 0\\
0 & (-2-h) & 4 & (h - 2) &\\
\vdots & & & \ddots\\
\\
\\
& & & 0 & (-2-h) & 4 & (h - 2)\\
& & & & 0 & 0 & 1  
\end{bmatrix}
$$
\newpage
\ \\
d)\\
\ \\
Since $h \sum_{j=1}^n v_j = \int_0^1 u(x_j)$ we can write. (same as in 3b)
$$
h \sum_{j=1}^n v_j^2 \leq \frac{1}{2} h \sum_{j=1}^n f(x_j) v_j
$$
Then use Cauchy-Schwartz.
$$
h \sum_{j=1}^n v_j^2 \leq \frac{1}{2} \left( h \sum_{j=1}^n f(x_j)^2 \right)^{\frac{1}{2}} 
\left( h \sum_{j=1}^n v_j^2 \right)^{\frac{1}{2}}
$$
Then use $|ab| \leq \frac{a^2}{2} + \frac{b^2}{2}$
$$
h \sum_{j=1}^n v_j^2 \leq \frac{1}{2} \left( \frac{1}{2} h \sum_{j=1}^n f(x_j)^2 \right)
\left( \frac{1}{2} h \sum_{j=1}^n v_j^2 \right)
$$
$$
4 h \sum_{j=1}^n v_j^2 - h \sum_{j=1}^n v_j^2 \leq h \sum_{j=1}^n f(x_j)^2
$$
$$
h \sum_{j=1}^n v_j^2 \leq \frac{1}{3} h \sum_{j=1}^n f(x_j)^2
$$
\newpage
\ \\
Question 5)\\
a)\\
\ \\
Program was implemented in python 3:
\begin{verbatim}
import numpy as np
import matplotlib.pyplot as plt
from scipy.linalg import lstsq

def f(x):
    if x == 0 or x == 1:
        return 0
    else:
        return 1

def u(x):
    return ((np.exp(x) - 1) / (1 - np.exp(1))) + x


n = 7
h = 1.0 / (1 + n) 
x = np.linspace(0, 1, n)
b = np.array([f(i)*2*(h**2) for i in x])

A = np.zeros((n,n))
A[0][0] = 1; A[-1][-1] = 1
for i in range(1, n-1):
    A[i][i-1] = - 2 - h
    A[i][i] = 4
    A[i][i+1] = h - 2

p, res, rnk, s = lstsq(A, b)

plt.plot(x, u(x), label="u(x)")
plt.plot(x, p, label="v")
plt.legend()
plt.show()
\end{verbatim}
\newpage
\ \\
Plot given by program for $n=7$\\
\includegraphics[scale=0.8]{"Figure_1.png"}
\ \\
b)\\
\ \\
Was not able to complete this task.
\end{document}